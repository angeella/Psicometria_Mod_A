% Options for packages loaded elsewhere
\PassOptionsToPackage{unicode}{hyperref}
\PassOptionsToPackage{hyphens}{url}
%
\documentclass[
  ignorenonframetext,
]{beamer}
\usepackage{pgfpages}
\setbeamertemplate{caption}[numbered]
\setbeamertemplate{caption label separator}{: }
\setbeamercolor{caption name}{fg=normal text.fg}
\beamertemplatenavigationsymbolsempty
% Prevent slide breaks in the middle of a paragraph
\widowpenalties 1 10000
\raggedbottom
\setbeamertemplate{part page}{
  \centering
  \begin{beamercolorbox}[sep=16pt,center]{part title}
    \usebeamerfont{part title}\insertpart\par
  \end{beamercolorbox}
}
\setbeamertemplate{section page}{
  \centering
  \begin{beamercolorbox}[sep=12pt,center]{part title}
    \usebeamerfont{section title}\insertsection\par
  \end{beamercolorbox}
}
\setbeamertemplate{subsection page}{
  \centering
  \begin{beamercolorbox}[sep=8pt,center]{part title}
    \usebeamerfont{subsection title}\insertsubsection\par
  \end{beamercolorbox}
}
\AtBeginPart{
  \frame{\partpage}
}
\AtBeginSection{
  \ifbibliography
  \else
    \frame{\sectionpage}
  \fi
}
\AtBeginSubsection{
  \frame{\subsectionpage}
}
\usepackage{lmodern}
\usepackage{amssymb,amsmath}
\usepackage{ifxetex,ifluatex}
\ifnum 0\ifxetex 1\fi\ifluatex 1\fi=0 % if pdftex
  \usepackage[T1]{fontenc}
  \usepackage[utf8]{inputenc}
  \usepackage{textcomp} % provide euro and other symbols
\else % if luatex or xetex
  \usepackage{unicode-math}
  \defaultfontfeatures{Scale=MatchLowercase}
  \defaultfontfeatures[\rmfamily]{Ligatures=TeX,Scale=1}
\fi
% Use upquote if available, for straight quotes in verbatim environments
\IfFileExists{upquote.sty}{\usepackage{upquote}}{}
\IfFileExists{microtype.sty}{% use microtype if available
  \usepackage[]{microtype}
  \UseMicrotypeSet[protrusion]{basicmath} % disable protrusion for tt fonts
}{}
\makeatletter
\@ifundefined{KOMAClassName}{% if non-KOMA class
  \IfFileExists{parskip.sty}{%
    \usepackage{parskip}
  }{% else
    \setlength{\parindent}{0pt}
    \setlength{\parskip}{6pt plus 2pt minus 1pt}}
}{% if KOMA class
  \KOMAoptions{parskip=half}}
\makeatother
\usepackage{xcolor}
\IfFileExists{xurl.sty}{\usepackage{xurl}}{} % add URL line breaks if available
\IfFileExists{bookmark.sty}{\usepackage{bookmark}}{\usepackage{hyperref}}
\hypersetup{
  pdftitle={Introduzione a R},
  pdfauthor={Angela Andreella},
  hidelinks,
  pdfcreator={LaTeX via pandoc}}
\urlstyle{same} % disable monospaced font for URLs
\newif\ifbibliography
\usepackage{color}
\usepackage{fancyvrb}
\newcommand{\VerbBar}{|}
\newcommand{\VERB}{\Verb[commandchars=\\\{\}]}
\DefineVerbatimEnvironment{Highlighting}{Verbatim}{commandchars=\\\{\}}
% Add ',fontsize=\small' for more characters per line
\usepackage{framed}
\definecolor{shadecolor}{RGB}{248,248,248}
\newenvironment{Shaded}{\begin{snugshade}}{\end{snugshade}}
\newcommand{\AlertTok}[1]{\textcolor[rgb]{0.94,0.16,0.16}{#1}}
\newcommand{\AnnotationTok}[1]{\textcolor[rgb]{0.56,0.35,0.01}{\textbf{\textit{#1}}}}
\newcommand{\AttributeTok}[1]{\textcolor[rgb]{0.77,0.63,0.00}{#1}}
\newcommand{\BaseNTok}[1]{\textcolor[rgb]{0.00,0.00,0.81}{#1}}
\newcommand{\BuiltInTok}[1]{#1}
\newcommand{\CharTok}[1]{\textcolor[rgb]{0.31,0.60,0.02}{#1}}
\newcommand{\CommentTok}[1]{\textcolor[rgb]{0.56,0.35,0.01}{\textit{#1}}}
\newcommand{\CommentVarTok}[1]{\textcolor[rgb]{0.56,0.35,0.01}{\textbf{\textit{#1}}}}
\newcommand{\ConstantTok}[1]{\textcolor[rgb]{0.00,0.00,0.00}{#1}}
\newcommand{\ControlFlowTok}[1]{\textcolor[rgb]{0.13,0.29,0.53}{\textbf{#1}}}
\newcommand{\DataTypeTok}[1]{\textcolor[rgb]{0.13,0.29,0.53}{#1}}
\newcommand{\DecValTok}[1]{\textcolor[rgb]{0.00,0.00,0.81}{#1}}
\newcommand{\DocumentationTok}[1]{\textcolor[rgb]{0.56,0.35,0.01}{\textbf{\textit{#1}}}}
\newcommand{\ErrorTok}[1]{\textcolor[rgb]{0.64,0.00,0.00}{\textbf{#1}}}
\newcommand{\ExtensionTok}[1]{#1}
\newcommand{\FloatTok}[1]{\textcolor[rgb]{0.00,0.00,0.81}{#1}}
\newcommand{\FunctionTok}[1]{\textcolor[rgb]{0.00,0.00,0.00}{#1}}
\newcommand{\ImportTok}[1]{#1}
\newcommand{\InformationTok}[1]{\textcolor[rgb]{0.56,0.35,0.01}{\textbf{\textit{#1}}}}
\newcommand{\KeywordTok}[1]{\textcolor[rgb]{0.13,0.29,0.53}{\textbf{#1}}}
\newcommand{\NormalTok}[1]{#1}
\newcommand{\OperatorTok}[1]{\textcolor[rgb]{0.81,0.36,0.00}{\textbf{#1}}}
\newcommand{\OtherTok}[1]{\textcolor[rgb]{0.56,0.35,0.01}{#1}}
\newcommand{\PreprocessorTok}[1]{\textcolor[rgb]{0.56,0.35,0.01}{\textit{#1}}}
\newcommand{\RegionMarkerTok}[1]{#1}
\newcommand{\SpecialCharTok}[1]{\textcolor[rgb]{0.00,0.00,0.00}{#1}}
\newcommand{\SpecialStringTok}[1]{\textcolor[rgb]{0.31,0.60,0.02}{#1}}
\newcommand{\StringTok}[1]{\textcolor[rgb]{0.31,0.60,0.02}{#1}}
\newcommand{\VariableTok}[1]{\textcolor[rgb]{0.00,0.00,0.00}{#1}}
\newcommand{\VerbatimStringTok}[1]{\textcolor[rgb]{0.31,0.60,0.02}{#1}}
\newcommand{\WarningTok}[1]{\textcolor[rgb]{0.56,0.35,0.01}{\textbf{\textit{#1}}}}
\usepackage{graphicx}
\makeatletter
\def\maxwidth{\ifdim\Gin@nat@width>\linewidth\linewidth\else\Gin@nat@width\fi}
\def\maxheight{\ifdim\Gin@nat@height>\textheight\textheight\else\Gin@nat@height\fi}
\makeatother
% Scale images if necessary, so that they will not overflow the page
% margins by default, and it is still possible to overwrite the defaults
% using explicit options in \includegraphics[width, height, ...]{}
\setkeys{Gin}{width=\maxwidth,height=\maxheight,keepaspectratio}
% Set default figure placement to htbp
\makeatletter
\def\fps@figure{htbp}
\makeatother
\setlength{\emergencystretch}{3em} % prevent overfull lines
\providecommand{\tightlist}{%
  \setlength{\itemsep}{0pt}\setlength{\parskip}{0pt}}
\setcounter{secnumdepth}{-\maxdimen} % remove section numbering

\title{Introduzione a R}
\subtitle{\ldots rapida e indolore}
\author{Angela Andreella}
\date{13/10/2020}
\logo{\includegraphics{Images/unipd\_logo.png}}

\begin{document}
\frame{\titlepage}

\hypertarget{lets-start}{%
\section{Let's start!}\label{lets-start}}

\begin{frame}{Conosciamoci e conosciamo R}
\protect\hypertarget{conosciamoci-e-conosciamo-r}{}
\begin{itemize}
\item
  Quanti di voi hanno \textbf{installato} R?
\item
  Quanti di voi hanno mai \textbf{usato} R o un altro linguaggio di
  programmazione?
\end{itemize}

R é un grande mondo, con grandi idee e iniziative!!
\includegraphics[width=0.1\textwidth,height=\textheight]{Images/R-LadiesGlobal.png}

\begin{itemize}
\item
  \textbf{Analisi riproducibili}
\item
  \textbf{Free}
\item
  In continua \textbf{evoluzione} ed espansione

  dunque..
  \includegraphics[width=0.1\textwidth,height=\textheight]{Images/download.png}
\end{itemize}
\end{frame}

\begin{frame}{Conosciamoci e conosciamo R}
\protect\hypertarget{conosciamoci-e-conosciamo-r-1}{}
Prima di tutto\ldots.

\textbf{Aprire Rstudio}

Abbiamo 4 finestrelle:

\begin{enumerate}
\item
  Scrittura Codice (\textbf{Script})
\item
  Esecuzione del Codice (\textbf{Console})
\item
  Files, Plots, \textbf{Help}
\item
  Visualizzazione \textbf{datasets}, oggetti caricati in generale, e
  \textbf{storia}
\end{enumerate}
\end{frame}

\begin{frame}[fragile]{Pacchetti per tutti!}
\protect\hypertarget{pacchetti-per-tutti}{}
R fornisce una serie di pacchetti (packages) che racchiudono funzioni
molto utili e di vario genere. Per esempio con i seguenti comandi:

\begin{Shaded}
\begin{Highlighting}[]
\KeywordTok{install.packages}\NormalTok{(}\StringTok{"xlsx"}\NormalTok{) }\CommentTok{\#Installiamo il pacchetto "xlsx"}
\KeywordTok{library}\NormalTok{(xlsx) }\CommentTok{\#Dopo averlo installato lo richiamiamo }
\end{Highlighting}
\end{Shaded}

installiamo il pacchetto \texttt{xlsx} e lo richiamiamo nella nostra
sessione. Questo pacchetto ci permette di caricare all'interno di R dati
da Excel.

\(\rightarrow\) per ogni cosa che volete fare esiste un pacchetto
corrispondente! Se trovate questo errore in futuro:

\begin{Shaded}
\begin{Highlighting}[]
\NormalTok{Error }\ControlFlowTok{in} \KeywordTok{library}\NormalTok{(dplyr) }\OperatorTok{:}\StringTok{ }\NormalTok{there is no package called ‘dplyr’}
\end{Highlighting}
\end{Shaded}

significa che dovete installare il pacchetto \texttt{dplyr}. Consiglio:
errori/bug cercate su google o
\href{https://stackoverflow.com}{stackoverflow}!
\end{frame}

\hypertarget{un-po-di-basi}{%
\section{Un po' di basi}\label{un-po-di-basi}}

\begin{frame}[fragile]{Operazioni}
\protect\hypertarget{operazioni}{}
Per le operazioni di somma, sottrazione, moltiplicazione e divisione si
utilizzano \(+ - * /\). Per l'elevazione a potenza si usa: \(**\) o
\(ˆ\).

\begin{Shaded}
\begin{Highlighting}[]
\NormalTok{(}\DecValTok{4}\OperatorTok{+}\DecValTok{5}\NormalTok{)}\OperatorTok{*}\DecValTok{3}
\end{Highlighting}
\end{Shaded}

Gli operatori logici corrispondono invece ai seguenti simboli:

\begin{Shaded}
\begin{Highlighting}[]
\OperatorTok{\&}\StringTok{ }\CommentTok{\#AND}
\ErrorTok{|}\StringTok{ }\CommentTok{\#OR}
\ErrorTok{\textgreater{}}\StringTok{ }\CommentTok{\#maggiore}
\ErrorTok{\textless{}}\StringTok{ }\CommentTok{\#minore}
\ErrorTok{==}\StringTok{ }\CommentTok{\#uguale}
\ErrorTok{!=}\StringTok{ }\CommentTok{\#diverso}
\ErrorTok{\textgreater{}=}\StringTok{ }\CommentTok{\#maggiore uguale}
\ErrorTok{\textless{}=}\StringTok{ }\CommentTok{\#minore uguale}
\end{Highlighting}
\end{Shaded}

Attenzione! \texttt{=} e \texttt{\textless{}-} si usano per assegnare un
valore a una variabile

\begin{Shaded}
\begin{Highlighting}[]
\NormalTok{pluto =}\StringTok{ "ciao"}
\end{Highlighting}
\end{Shaded}

diverso da \texttt{==}!! 💥
\end{frame}

\begin{frame}[fragile]{Operazioni}
\protect\hypertarget{operazioni-1}{}
Vediamo qualche esempio insieme

\begin{Shaded}
\begin{Highlighting}[]
\DecValTok{4}\OperatorTok{==}\DecValTok{10}
\DecValTok{5}\OperatorTok{!=}\DecValTok{10}
\NormalTok{(}\DecValTok{4}\OperatorTok{==}\DecValTok{4}\NormalTok{)}\OperatorTok{\&}\NormalTok{(}\DecValTok{10}\OperatorTok{!=}\DecValTok{1}\NormalTok{)}
\DecValTok{8}\OperatorTok{\textgreater{}=}\DecValTok{10}
\DecValTok{3}\OperatorTok{\textless{}}\DecValTok{20}
\end{Highlighting}
\end{Shaded}

ricordiamo che:

\begin{Shaded}
\begin{Highlighting}[]
\OtherTok{TRUE}\OperatorTok{==}\DecValTok{1}
\end{Highlighting}
\end{Shaded}

\begin{Shaded}
\begin{Highlighting}[]
\OtherTok{FALSE}\OperatorTok{==}\DecValTok{0}
\end{Highlighting}
\end{Shaded}
\end{frame}

\begin{frame}[fragile]{Altri comandi fondamentali}
\protect\hypertarget{altri-comandi-fondamentali}{}
Aiuto?

\begin{Shaded}
\begin{Highlighting}[]
\KeywordTok{help}\NormalTok{(}\StringTok{"+"}\NormalTok{) }\CommentTok{\#cerchiamo aiuto per capire l\textquotesingle{}uso di +}
\end{Highlighting}
\end{Shaded}

Cambiare working directory?

\begin{Shaded}
\begin{Highlighting}[]
\KeywordTok{setwd}\NormalTok{(}\StringTok{"il\_mio\_path"}\NormalTok{)}
\end{Highlighting}
\end{Shaded}

Vedere cosa c'é nella mia working directory?

\begin{Shaded}
\begin{Highlighting}[]
\KeywordTok{ls}\NormalTok{()}
\end{Highlighting}
\end{Shaded}

Cancellare quello che abbiamo trovato nella working directory?

\begin{Shaded}
\begin{Highlighting}[]
\KeywordTok{rm}\NormalTok{(}\DataTypeTok{list =} \KeywordTok{ls}\NormalTok{()) }\CommentTok{\#Cancelliamo tutti gli elementi presenti}
\KeywordTok{rm}\NormalTok{(pluto) }\CommentTok{\#cancelliamo l\textquotesingle{}oggetto pluto}
\end{Highlighting}
\end{Shaded}
\end{frame}

\begin{frame}[fragile]{Caricamento dati}
\protect\hypertarget{caricamento-dati}{}
Come caricare i dati? dipende\ldots{} 😼

Per prima cosa vedete di che formato é il vostro documento (xlsx, csv,
xls, txt etc), in generale:

\begin{Shaded}
\begin{Highlighting}[]
\KeywordTok{read.table}\NormalTok{(}\StringTok{"path/file\_name.txt"}\NormalTok{) }\CommentTok{\#formato txt}
\KeywordTok{read.csv}\NormalTok{(}\StringTok{"path/file\_name.csv"}\NormalTok{) }\CommentTok{\#formato csv}
\KeywordTok{read.xlsx}\NormalTok{(}\StringTok{"path/file\_name.xlsx"}\NormalTok{) }\CommentTok{\#formato xlsx}
\end{Highlighting}
\end{Shaded}

etc.. se siete indecisi usate l help o google! 😄

Proviamo insieme!

Qualche funzione utile:

\begin{Shaded}
\begin{Highlighting}[]
\KeywordTok{colnames}\NormalTok{(db) }
\KeywordTok{rownames}\NormalTok{(db)}
\end{Highlighting}
\end{Shaded}
\end{frame}

\hypertarget{ma-le-variabili}{%
\section{Ma le variabili?}\label{ma-le-variabili}}

\begin{frame}[fragile]{Tipologie di variabili}
\protect\hypertarget{tipologie-di-variabili}{}
Carichiamo il file di esempio:

\begin{Shaded}
\begin{Highlighting}[]
\NormalTok{DB \textless{}{-}}\StringTok{ }\KeywordTok{read.csv}\NormalTok{(}\StringTok{"Datasets/HappinessAlcoholConsumption.csv"}\NormalTok{)}
\end{Highlighting}
\end{Shaded}

e vediamo qualche esempio di variabile:

\begin{Shaded}
\begin{Highlighting}[]
\KeywordTok{str}\NormalTok{(DB) }\CommentTok{\#con questo comando vediamo la struttura del dataset DB}
\KeywordTok{head}\NormalTok{(DB) }\CommentTok{\#Vediamo le prime 6 righe/osservazioni}
\KeywordTok{View}\NormalTok{(DB) }\CommentTok{\#Vediamo in finestra separata il dataset intero}
\end{Highlighting}
\end{Shaded}

\begin{enumerate}
\item
  Numeric : \texttt{HappinessScore}
\item
  Factor (Variabili Qualitative): \texttt{Country}
\item
  Integer: \texttt{Beer\_PerCapita}
\end{enumerate}

ma abbiamo anche character, logical, \texttt{NA}, \texttt{NaN}.
\end{frame}

\begin{frame}[fragile]{Tipologie di oggetti (alcuni)}
\protect\hypertarget{tipologie-di-oggetti-alcuni}{}
Vettori

\begin{Shaded}
\begin{Highlighting}[]
\NormalTok{pluto \textless{}{-}}\StringTok{ }\KeywordTok{c}\NormalTok{(}\DecValTok{1}\OperatorTok{:}\DecValTok{100}\NormalTok{) }\CommentTok{\#creaiamo un vettore con valori da 1 a 100}
\NormalTok{pluto1 \textless{}{-}}\StringTok{ }\KeywordTok{seq}\NormalTok{(}\DecValTok{1}\NormalTok{,}\DecValTok{500}\NormalTok{,}\DecValTok{5}\NormalTok{) }\CommentTok{\#creaiamo un vettore con valori da 1 a 500 ogni  5 valori}
\NormalTok{pluto2 \textless{}{-}}\StringTok{ }\KeywordTok{rep}\NormalTok{(}\DecValTok{10}\NormalTok{,}\DecValTok{5}\NormalTok{) }\CommentTok{\#creiamo un vettore dove si ripete il valore 10 5 volte}
\NormalTok{pluto[}\DecValTok{3}\NormalTok{] }\CommentTok{\#richiamiamo l\textquotesingle{}elemento del vettore pluto che sta nella 3 posizione}
\NormalTok{pluto[}\DecValTok{1}\OperatorTok{:}\DecValTok{3}\NormalTok{] }\CommentTok{\#richiamiamo l\textquotesingle{}elemento del vettore pluto che sta nella 1,2,3 posizioni}
\KeywordTok{is.vector}\NormalTok{(pluto) }\CommentTok{\#verifichiamo che pluto sia un vettore}
\KeywordTok{length}\NormalTok{(pluto) }\CommentTok{\#dimensione del vettore pluto?}
\KeywordTok{sort}\NormalTok{(pluto) }\CommentTok{\#ordiniamo pluto}
\NormalTok{pluto}\OperatorTok{/}\DecValTok{5} \CommentTok{\#dividiamo ogni elemento di pluto per 5}
\end{Highlighting}
\end{Shaded}

proviamo insieme!
\end{frame}

\begin{frame}[fragile]{Tipologie di oggetti (alcuni)}
\protect\hypertarget{tipologie-di-oggetti-alcuni-1}{}
Matrici

\begin{Shaded}
\begin{Highlighting}[]
\NormalTok{pippo \textless{}{-}}\StringTok{ }\KeywordTok{matrix}\NormalTok{(}\DataTypeTok{data =} \DecValTok{10}\NormalTok{, }\DataTypeTok{nrow =} \DecValTok{10}\NormalTok{, }\DataTypeTok{ncol =} \DecValTok{5}\NormalTok{) }\CommentTok{\#Creiamo una matrice }
\CommentTok{\#con 10 righe e 5 colonne contenente solo 10}
\NormalTok{pippo1 \textless{}{-}}\StringTok{ }\KeywordTok{matrix}\NormalTok{(}\DataTypeTok{data =}\NormalTok{ pluto, }\DataTypeTok{nrow =} \DecValTok{20}\NormalTok{, }\DataTypeTok{ncol =} \DecValTok{5}\NormalTok{, }\DataTypeTok{byrow =}\NormalTok{ T) }\CommentTok{\#mettiamo pluto in}
\CommentTok{\#pippo! partendo dalle righe (byrow)}
\NormalTok{pippo2 \textless{}{-}}\StringTok{ }\KeywordTok{cbind}\NormalTok{(pluto,pluto1) }\CommentTok{\#Uniamo i due vettori per colonna}
\NormalTok{pippo3 \textless{}{-}}\StringTok{ }\KeywordTok{rbind}\NormalTok{(pluto,pluto1) }\CommentTok{\#Uniamo i due vettori per riga}
\KeywordTok{dim}\NormalTok{(pippo) }\CommentTok{\#Dimensione di pippo?}
\KeywordTok{ncol}\NormalTok{(pippo) }\CommentTok{\#numero di colonne di pippo?}
\KeywordTok{nrow}\NormalTok{(pippo) }\CommentTok{\#numero di righe di pippo?}
\KeywordTok{is.matrix}\NormalTok{(pippo) }\CommentTok{\#Check che pippo sia una matrice }
\end{Highlighting}
\end{Shaded}

se non vi ricordate \(\rightarrow\) \texttt{help(matrix)}! 💥
\end{frame}

\begin{frame}[fragile]{Tipologie di oggetti (alcuni)}
\protect\hypertarget{tipologie-di-oggetti-alcuni-2}{}
Data frame

\begin{Shaded}
\begin{Highlighting}[]
\KeywordTok{is.data.frame}\NormalTok{(DB)}
\NormalTok{rick \textless{}{-}}\StringTok{ }\KeywordTok{data.frame}\NormalTok{(pippo2[}\DecValTok{1}\OperatorTok{:}\DecValTok{20}\NormalTok{,]) }\CommentTok{\#creiamo un dataframe di nome rick }
\CommentTok{\#che prende le prime 20 osservazioni di pippo2}
\KeywordTok{dim}\NormalTok{(rick) }\CommentTok{\#Dimensione? nrow, ncol etc come matrix}
\KeywordTok{colnames}\NormalTok{(rick) }\CommentTok{\#Nomi colonna?}
\KeywordTok{rownames}\NormalTok{(rick) }\CommentTok{\#Nomi riga?}
\KeywordTok{rownames}\NormalTok{(rick) \textless{}{-}}\StringTok{ }\NormalTok{letters[}\DecValTok{1}\OperatorTok{:}\KeywordTok{dim}\NormalTok{(rick)[}\DecValTok{1}\NormalTok{]] }\CommentTok{\#rinominiamo le righe di rick }
\CommentTok{\#come le prime 20 lettere dell\textquotesingle{}alfabeto}
\NormalTok{rick[}\DecValTok{10}\NormalTok{,}\DecValTok{2}\NormalTok{] }\CommentTok{\#Richiamiamo l\textquotesingle{}elemento della 10 riga e 2 colonna}
\NormalTok{rick}\OperatorTok{$}\NormalTok{pluto }\CommentTok{\#richiamo la variabile pluto (prima colonna)}
\end{Highlighting}
\end{Shaded}

Che differenza c'é tra \texttt{data.frame} e \texttt{matrix}? ❓
\end{frame}

\hypertarget{prima-di-continuare..}{%
\section{Prima di continuare..}\label{prima-di-continuare..}}

\begin{frame}[fragile]{}
\protect\hypertarget{section}{}
\begin{enumerate}
\item
  Domande ❓
\item
  Ma perdo tutto quello che ho caricato/scritto?
\end{enumerate}

\begin{Shaded}
\begin{Highlighting}[]
\KeywordTok{save}\NormalTok{(rick, }\DataTypeTok{file =} \StringTok{"your\_path/il\_mio\_primo\_rda.rda"}\NormalTok{) }\CommentTok{\#salviamo l\textquotesingle{}oggetto }
\CommentTok{\#rick in il\_mio\_primo\_rda}
\KeywordTok{load}\NormalTok{(}\StringTok{"your\_path/il\_mio\_primo\_rda.rda"}\NormalTok{) }\CommentTok{\#e lo ricarico sulla working directory!!}
\end{Highlighting}
\end{Shaded}
\end{frame}

\hypertarget{un-po-di-analisi-descrittiva}{%
\section{Un po' di analisi
descrittiva}\label{un-po-di-analisi-descrittiva}}

\begin{frame}[fragile]{Variabili qualitative}
\protect\hypertarget{variabili-qualitative}{}
\begin{block}{Tabelle di Frequenza}
\protect\hypertarget{tabelle-di-frequenza}{}
Frequenze assolute

\begin{Shaded}
\begin{Highlighting}[]
\NormalTok{fa\_h \textless{}{-}}\StringTok{ }\KeywordTok{table}\NormalTok{(DB}\OperatorTok{$}\NormalTok{Hemisphere)}
\KeywordTok{addmargins}\NormalTok{(fa\_h) }\CommentTok{\#Aggiungi totale alla tabella fa\_reg}
\end{Highlighting}
\end{Shaded}

\begin{verbatim}
## 
##  both north  noth south   Sum 
##     5    92     4    21   122
\end{verbatim}
\end{block}
\end{frame}

\begin{frame}[fragile]{Variabili qualitative}
\protect\hypertarget{variabili-qualitative-1}{}
\begin{block}{Tabelle di Frequenza}
\protect\hypertarget{tabelle-di-frequenza-1}{}
Frequenze relative

\begin{Shaded}
\begin{Highlighting}[]
\NormalTok{fa\_h}\OperatorTok{/}\KeywordTok{nrow}\NormalTok{(DB)}
\end{Highlighting}
\end{Shaded}

\begin{verbatim}
## 
##       both      north       noth      south 
## 0.04098361 0.75409836 0.03278689 0.17213115
\end{verbatim}

\begin{Shaded}
\begin{Highlighting}[]
\NormalTok{fr\_h \textless{}{-}}\StringTok{ }\KeywordTok{prop.table}\NormalTok{(fa\_h)}
\KeywordTok{addmargins}\NormalTok{(fr\_h) }\CommentTok{\#Aggiungi totale alla tabella fr\_reg}
\end{Highlighting}
\end{Shaded}

\begin{verbatim}
## 
##       both      north       noth      south        Sum 
## 0.04098361 0.75409836 0.03278689 0.17213115 1.00000000
\end{verbatim}
\end{block}
\end{frame}

\begin{frame}{Variabili qualitative}
\protect\hypertarget{variabili-qualitative-2}{}
\begin{block}{Grafici}
\protect\hypertarget{grafici}{}
\includegraphics[width=1\textwidth,height=\textheight]{Images/pie1.gif}
\end{block}
\end{frame}

\begin{frame}[fragile]{Variabili qualitative}
\protect\hypertarget{variabili-qualitative-3}{}
\begin{block}{Grafici}
\protect\hypertarget{grafici-1}{}
Barplot

\begin{Shaded}
\begin{Highlighting}[]
\KeywordTok{barplot}\NormalTok{(fa\_h)}
\end{Highlighting}
\end{Shaded}

\includegraphics{IntRoduzione_a_R_files/figure-beamer/unnamed-chunk-23-1.pdf}
\end{block}
\end{frame}

\begin{frame}[fragile]{Variabili qualitative}
\protect\hypertarget{variabili-qualitative-4}{}
\begin{block}{Grafici}
\protect\hypertarget{grafici-2}{}
Grafico a torta

\begin{Shaded}
\begin{Highlighting}[]
\KeywordTok{pie}\NormalTok{(fa\_h)}
\end{Highlighting}
\end{Shaded}

\includegraphics{IntRoduzione_a_R_files/figure-beamer/unnamed-chunk-24-1.pdf}
\end{block}
\end{frame}

\begin{frame}[fragile]{Variabili quantitative}
\protect\hypertarget{variabili-quantitative}{}
\begin{block}{Tabelle di Frequenza e altro}
\protect\hypertarget{tabelle-di-frequenza-e-altro}{}
Per le tabelle di frequenza assolute e relative valgono gli stessi
comandi usati per la variabile qualitativa \texttt{Hemisphere}.

Con le variabili quantitative abbiamo anche:

\begin{Shaded}
\begin{Highlighting}[]
\KeywordTok{min}\NormalTok{(DB}\OperatorTok{$}\NormalTok{HappinessScore)}
\end{Highlighting}
\end{Shaded}

\begin{verbatim}
## [1] 3.069
\end{verbatim}

\begin{Shaded}
\begin{Highlighting}[]
\KeywordTok{max}\NormalTok{(DB}\OperatorTok{$}\NormalTok{HappinessScore)}
\end{Highlighting}
\end{Shaded}

\begin{verbatim}
## [1] 7.526
\end{verbatim}

\begin{Shaded}
\begin{Highlighting}[]
\KeywordTok{summary}\NormalTok{(DB}\OperatorTok{$}\NormalTok{HappinessScore)}
\end{Highlighting}
\end{Shaded}

\begin{verbatim}
##    Min. 1st Qu.  Median    Mean 3rd Qu.    Max. 
##   3.069   4.528   5.542   5.525   6.477   7.526
\end{verbatim}

\begin{Shaded}
\begin{Highlighting}[]
\KeywordTok{mean}\NormalTok{(DB}\OperatorTok{$}\NormalTok{HappinessScore)}
\end{Highlighting}
\end{Shaded}

\begin{verbatim}
## [1] 5.524828
\end{verbatim}

\begin{Shaded}
\begin{Highlighting}[]
\KeywordTok{median}\NormalTok{(DB}\OperatorTok{$}\NormalTok{HappinessScore)}
\end{Highlighting}
\end{Shaded}

\begin{verbatim}
## [1] 5.542
\end{verbatim}
\end{block}
\end{frame}

\begin{frame}[fragile]{Variabili quantitative}
\protect\hypertarget{variabili-quantitative-1}{}
\begin{block}{Grafici}
\protect\hypertarget{grafici-3}{}
Istogramma

\begin{Shaded}
\begin{Highlighting}[]
\KeywordTok{hist}\NormalTok{(DB}\OperatorTok{$}\NormalTok{HappinessScore,}\DataTypeTok{col =} \StringTok{"blue"}\NormalTok{, }\DataTypeTok{main =} \StringTok{"Happiness Score"}\NormalTok{)}
\KeywordTok{hist}\NormalTok{(DB}\OperatorTok{$}\NormalTok{HappinessScore, }\DataTypeTok{breaks =} \DecValTok{20}\NormalTok{) }\CommentTok{\#con breaks decidi gli intervalli!}
\end{Highlighting}
\end{Shaded}

Boxplot

\begin{Shaded}
\begin{Highlighting}[]
\KeywordTok{boxplot}\NormalTok{(DB}\OperatorTok{$}\NormalTok{HappinessScore,}\DataTypeTok{col =} \StringTok{"blue"}\NormalTok{)}
\end{Highlighting}
\end{Shaded}
\end{block}
\end{frame}

\end{document}
